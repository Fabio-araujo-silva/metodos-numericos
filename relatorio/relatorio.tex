\documentclass{article} % change to article if necessary 
\usepackage[brazilian]{babel}
\usepackage{graphicx, txfonts} % Required for inserting images
\usepackage{amsfonts}
\usepackage{amsmath}
\usepackage{amssymb}
\usepackage{amsthm} \usepackage{mathtools} \usepackage{lmodern} \usepackage{tikz}
\usepackage{cancel}
\usepackage{booktabs}
\usepackage{tabularx}
\usepackage[shortlabels]{enumitem}
\newcommand\myalign[1]{\centerline{$\displaystyle#1$}}
\usepackage{tcolorbox} \tcbuselibrary{theorems}
\newcommand{\heart}
{\ensuremath\heartsuit}
\newcommand{\mdc}{\mathrm{mdc}}
\DeclareMathOperator{\diam}{diam}
\DeclareMathOperator{\rot}{rot}
\DeclareMathOperator{\gr}{Gr}
\DeclareMathOperator{\tr}{tr}
\newcommand{\filledheart}{\ensuremath\varheartsuit}

\title{Relatório – Sobre diferentes métodos de resolução de sistemas lineares para computar os índices de criminalidade sobre um grafo de Manhattan}
\author
{
  Vinícius Girão de Castro | 15491730 \\
  Natália Carvalho | 15497232 \\
  Larissa Rocha Gonçalves| 15522431\\
  Fabio Araujo | 16311045 \\
}
\date{}

\pdfsuppresswarningpagegroup=1

%\newtheorem{theorem}{Teorema}[section]
%\newtheorem{prop}{Proposição}[theorem]
\newtheorem{definition}{Definição}[section]
\newtheorem{exer}{}[section]
\newtheorem{prop}{Proposição}[section]
\newtheorem{cor}{Corolário}[prop]
\newtheorem{exemplo}{Exemplo}[section]
\newtheorem{resultado}{Resultado}[section]
\newtcbtheorem[number within=section]{theorem}{Teorema}%
{colback=red!5,colframe=red!35!black,fonttitle=\bfseries}{th}
\newtcbtheorem[number within=section]{propp}{Proposição}%
{colback=green!5,colframe=green!35!black,fonttitle=\bfseries}{pr}
\newtcbtheorem[number within=section]{lema}{Lema}%
{colback=yellow!5,colframe=yellow!35!black,fonttitle=\bfseries}{lm}

\theoremstyle{definition}
\newenvironment{dem}{\paragraph{\textit{Dem.:}}}{\hfill$\square$}

\begin{document}
\maketitle
    \section*{O que fizemos?}
        \begin{enumerate}
            \item Extraímos a maior componente conexa $G$ do grafo providenciado.
            \item Escolhemos, convenientemente, 10 vértices de $G$ e valores, entre 1 e 10, para cada um destes.
            \item Construímos a matriz laplaciana $L$ e a matriz de penalidades $P$, com $\alpha = 10^{7}$, assim como o vetor $b$, como instruído.
            \item Resolvemos o sistema  $\left( L + P \right)x = Pb $ por meio de diferentes métodos e comparamos os resultados obtidos.
        \end{enumerate}
    \section*{Para os métodos iterativos...    }
        \begin{itemize}
            \item Utilizamos uma tolerância de 1e-6.
            \item Empregamos a norma $\ell^2$ (euclidiana para vetores reais) para verificar a qualidade da convergência de cada método. Além disso, também calculamos a norma relativa da seguinte forma:
                $$\ell^2_{rel}(x) = \frac{\|x - y\|}{\|y\| + \epsilon }, \text{ com } \epsilon > 0,$$
                onde $y$ é a solução exata.
        \end{itemize}
    \section*{Métodos utilizados}
        \subsection*{Decomposição LU}
            \begin{itemize}
                \item Levou 792.882385 segundos (aproximadamente 13 minutos).
            \end{itemize}
        \subsection*{Decomposição de Cholesky}
            \begin{itemize}
                \item Como $G$ é um grafo conexo, sabemos que $L$ é simétrica semidefinida positiva.
                \item Levou 298.476189 segundos (aproximadamente 5 minutos).
            \end{itemize}
        \subsection*{Método de Jacobi}
            \begin{itemize}
                \item Convergiu com 1000 iterações e em 0.545334 segundo.
                \item Este método foi o que mais divergiu do resultado exato, com uma norma $\ell^2$ da diferença em relação ao resultado exato de 3.768525e+02 e uma norma relativa de 8.318139e-01, o que justifica o mapa insatisfatório.
            \end{itemize}
        \subsection*{Gauss-Seidel}
            \begin{itemize}
                \item Convergiu com 1000 iterações e em 0.368160 segundo.
                \item Apesar de ter se aproximado mais da solução exata do que o método de Jacobi, com uma norma $\ell^2$ da diferença de 3.274648e+02 e uma norma relativa de 7.228020e-01, o método de Gauss-Seidel ainda resulta numa representação longe da ideal.
            \end{itemize}

        \subsection*{Gradientes conjugados (CG)}
            \begin{itemize}
                \item A convergência do método dos gradientes conjugados é garantida pelo fato de que a matriz  $L + P$ é simétrica definida positiva.
                \item  O método CG convergiu com 1101 iterações e em 0.1703 segundo.
                \item Convergiu eficientemente, com uma norma $\mathcal{\ell}^2$ da diferença de 3.172346e-05 e uma norma relativa de 7.002212e-08.
            \end{itemize}

    \section*{Interpretando os resultados}
        \begin{itemize}
            \item \textbf{Métodos diretos:}

                Apesar de ambos terem fornecido a resposta exata, o método de Cholesky o fez num tempo consideravelmente menor.
            \item \textbf{Métodos iterativos:}
                
                Os métodos de Gauss-Jacobi e de Gauss-Seidel se provaram ineficazes para solucionar o sistema de forma aceitável, enquanto que o método dos gradientes conjugados foi extremamente eficiente.
        \end{itemize}

            Neste caso, convém optar pelo método dos gradientes conjugados!
            \end{document}
