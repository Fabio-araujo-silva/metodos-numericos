\documentclass{article} % change to article if necessary 
\usepackage[brazilian]{babel}
\usepackage{graphicx, txfonts} % Required for inserting images
\usepackage{amsfonts}
\usepackage{amsmath}
\usepackage{amssymb}
\usepackage{amsthm} \usepackage{mathtools} \usepackage{lmodern} \usepackage{tikz}
\usepackage{cancel}
\usepackage[shortlabels]{enumitem}
\newcommand\myalign[1]{\centerline{$\displaystyle#1$}}
\usepackage{tcolorbox} \tcbuselibrary{theorems}
\newcommand{\heart}
{\ensuremath\heartsuit}
\newcommand{\mdc}{\mathrm{mdc}}
\DeclareMathOperator{\diam}{diam}
\DeclareMathOperator{\rot}{rot}
\DeclareMathOperator{\gr}{Gr}
\DeclareMathOperator{\tr}{tr}
\newcommand{\filledheart}{\ensuremath\varheartsuit}

\title{Relatório – Solução de Sistemas Lineares Aplicados a Grafos de Manhattan}
\author{Vinícius Girão de Castro}
\date{}

\pdfsuppresswarningpagegroup=1

%\newtheorem{theorem}{Teorema}[section]
%\newtheorem{prop}{Proposição}[theorem]
\newtheorem{definition}{Definição}[section]
\newtheorem{exer}{}[section]
\newtheorem{prop}{Proposição}[section]
\newtheorem{cor}{Corolário}[prop]
\newtheorem{exemplo}{Exemplo}[section]
\newtheorem{resultado}{Resultado}[section]
\newtcbtheorem[number within=section]{theorem}{Teorema}%
{colback=red!5,colframe=red!35!black,fonttitle=\bfseries}{th}
\newtcbtheorem[number within=section]{propp}{Proposição}%
{colback=green!5,colframe=green!35!black,fonttitle=\bfseries}{pr}
\newtcbtheorem[number within=section]{lema}{Lema}%
{colback=yellow!5,colframe=yellow!35!black,fonttitle=\bfseries}{lm}

\theoremstyle{definition}
\newenvironment{dem}{\paragraph{\textit{Dem.:}}}{\hfill$\square$}

\begin{document}
\maketitle
    \section*{Objetivo}
        Este trabalho tem como objetivo resolver um sistema linear oriundo de um grafo de ruas da 
        ilha de Manhattan, utilizando diferentes métodos numéricos: LU, Cholesky,
        Jacobi, Gauss-Seidel e Gradientes Conjugados; e comparar o desempenho de cada um.
    \section*{Descrição do problema}
        Utilizando os arquivos manh.el (arestas) e manh.xy (coordenadas dos vértices),
        foi construído um grafo representando o sistema de ruas de Manhattan.
        As etapas seguidas foram:
        \begin{itemize}
            \item Seleção da maior componente conexa do grafo;
            \item escolha de $k$ vértices aleatórios $v_{i1}, v_{i2}, \ldots, v_{ik}$ e atribuição
                de valores $c_{i1}, c_{i2}, \ldots, c_{ik} \in (0, 10];$
            \item construção da matriz Laplaciana $L$ do grafo;
            \item Construção da matriz de penalidades  $P = \left(P_{ij}\right) $, onde  
                $$P_{ij} =
                \begin{cases}
                    \alpha = 1.0e7, \text{ se $j$ é um índice de um vértice escolhido} \\
                    0, \text{ caso contrário;}
                \end{cases}$$
            \item construção do vetor $b = \left( b_j \right)$, onde
                $$b_j= 
                \begin{cases}
                    c_{i_s}, \text{ se } $j=i_s$ \\
                    0, \text{ caso contrário;}
                \end{cases}$$
            \item resolução do sistema $\left(L+P \right)x=Pb$.
         \end{itemize}
    \section*{Métodos utilizados}
        \subsection{Decomposição LU}
            \begin{itemize}
                \item A matriz $A = L + P$ foi decomposta em  $A = LU$;
                \item Resolvido em duas etapas:
                    \begin{itemize}[a)]
                        \item $Ly = Pb$
                        \item  $Ux = y$;
                    \end{itemize}
                \item vantagem: aplicável à qualquer matriz não singular;
                \item desvantagem: alto custo para matrizes esparsas.
            \end{itemize}
        \subsection{Decomposição de Cholesky}
            
O método iterativo estrutura-se com a decomposição $A = HH^t$, sendo $H$ triangular inferior.
Então é calculada a solução em duas etapas:
\begin{itemize}[a)]
    \item $Hy = Pb$
    \item  $H^tx = y$.
\end{itemize}
Para que possamos determinar $H$, é necessário que a matriz $A$ seja simétrica definida positiva.

Sabe-se que toda matriz Laplaciana de um grafo conexo é necessariamente simétrica semidefinida
positiva, mas ao adicionar a matriz de penalidades, com valores grandes e positivos na diagonal,
algumas posições da solução são fortemente forçadas a assumir os valores desejados definidos em b.
Esses grandes valores diagonais empurram todos os autovalores da matriz para cima, garantindo
que todos sejam estritamente positivos.

Em conclusão, a matriz A torna-se simétrica definida positiva
(SPD), o que justifica o uso seguro da Decomposição de Cholesky.
Tal método tem como vantagem ser mais eficiente que LU, apesar de requerer que a matriz seja SPD.
Para o teste com 400 valores, o método levou $306.722567$ segundos, ou seja, pouco mais de 5 minutos.
\end{document}
